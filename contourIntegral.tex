\documentclass[12pt, a4paper]{article}
\usepackage[utf8]{inputenc}
\usepackage{mathtools}
\usepackage[english, russian]{babel}
\usepackage[T1, T2A]{fontenc}
\usepackage[intlimits]{amsmath}

\title{Вычисление интеграла}
\author{Гончар Дарья, 7 группа }
\date{}
\thispagestyle{empty}
\begin{document}

\textbf{Гончар Дарья, 7 группа}
\paragraph{}
Вычислить интеграл  \begin{equation*}
\oint\limits_C \frac{e^z}{(z-i)^2(z+2)}\, \mathrm{d} z \end{equation*}

в следующих случаях задания контура: а) |z\big-i|=2 ;   б) |z+2\big-i|=3
\paragraph{Решение:}
Найдем особые точки подынтегральной функции. Ими являются нули знаменателя: z=i, z=\big-2.
\paragraph{}
a) В круг контура |z\big-i|=2 попадает только точка z=i, вычисляем интеграл:
\begin{equation*} \oint\limits_C \frac{e^z}{(z-i)^2(z+2)}\, \mathrm{d} z = \int\limits_{|z-i|=2} \frac{\frac{e^z}{z+2}}{(z-i)^2}\, \mathrm{d} z  = 2\pi i\Bigg(\frac{e^z}{z+2}\Bigg)^{'} \Bigg|_{z=i} = 
\end{equation*} 
\begin{equation*}
   = 2\pi i\Bigg(\frac{e^z(z+2)-e^z}{(z+2)^2}\Bigg)\Bigg|_{z=i} = 2\pi i\frac{e^i(1+i)}{(2+i)^2} 
\end{equation*}

\paragraph{}
б) В круг контура |z+2\big-i|=3 попадают две точки z=i и z=\big-2. Поэтому находим интеграл как сумму двух интегралов по контурам, которые включают в себя только одну из двух точек. Вычисляем интеграл:
\begin{equation*} \oint\limits_C \frac{e^z}{(z-i)^2(z+2)}\, \mathrm{d} z = \int\limits_{C_1} \frac{e^z}{(z-i)^2(z+2)}\, \mathrm{d} z + \int\limits_{C_2} \frac{e^z}{(z-i)^2(z+2)}\, \mathrm{d} z=\end{equation*}
\begin{equation*}
\int\limits_{C_1} \frac{\frac{e^z}{z+2}}{(z-i)^2}\, \mathrm{d} z  + \int\limits_{C_2} \frac{\frac{e^z}{(z-i)^2}}{z+2}\, \mathrm{d} z = 2\pi i\Bigg(\frac{e^z}{z+2}\Bigg)^{'} \Bigg|_{z=i} + 2\pi i\Bigg(\frac{e^z}{(z-i)^2}\Bigg) \Bigg|_{z=-2} = 
\end{equation*} 
\paragraph{}
\begin{equation*}
   = 2\pi i\Bigg(\frac{e^z(z+2)-e^z}{(z+2)^2}\Bigg)\Bigg|_{z=i} +  2\pi i\frac{e^{\big-2}}{(\big-2\big-i)^2} = 2\pi i\frac{e^i(1+i)}{(2+i)^2} + 2\pi i\frac{e^{\big-2}}{(2+i)^2} = 
\end{equation*}
\paragraph{}
\begin{equation*}
  = \frac{2\pi i}{(2+i)^2} \Big(e^i(1+i)+e^{\big-2}\Big)
\end{equation*}
\end{document}